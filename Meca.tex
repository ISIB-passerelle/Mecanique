\documentclass[]{article}
\usepackage{lmodern}
\usepackage{amssymb,amsmath}
\usepackage{ifxetex,ifluatex}
\usepackage{fixltx2e} % provides \textsubscript
\usepackage{upgreek} % provides peta
\usepackage{gensymb} % provides degree
\usepackage{mathtools}
\usepackage[margin=2.5cm]{geometry}
\ifnum 0\ifxetex 1\fi\ifluatex 1\fi=0 % if pdftex
  \usepackage[T1]{fontenc}
  \usepackage[utf8]{inputenc}
\else % if luatex or xelatex
  \ifxetex
    \usepackage{mathspec}
  \else
    \usepackage{fontspec}
  \fi
  \defaultfontfeatures{Ligatures=TeX,Scale=MatchLowercase}
\fi
% use upquote if available, for straight quotes in verbatim environments
\IfFileExists{upquote.sty}{\usepackage{upquote}}{}
% use microtype if available
\IfFileExists{microtype.sty}{%
\usepackage{microtype}
\UseMicrotypeSet[protrusion]{basicmath} % disable protrusion for tt fonts
}{}
\usepackage{hyperref}
\hypersetup{unicode=true,
            pdfborder={0 0 0},
            breaklinks=true}
\urlstyle{same}  % don’t use monospace font for urls
%\setlength\footskip{10pt}
\IfFileExists{parskip.sty}{%
\usepackage{parskip}
}{% else
\setlength{\parindent}{0pt}
\setlength{\parskip}{6pt plus 2pt minus 1pt}
}
\setlength{\emergencystretch}{3em}  % prevent overfull lines
\providecommand{\tightlist}{%
  \setlength{\itemsep}{0pt}\setlength{\parskip}{0pt}}
\setcounter{secnumdepth}{0}
% Redefines (sub)paragraphs to behave more like sections
\ifx\paragraph\undefined\else
\let\oldparagraph\paragraph
\renewcommand{\paragraph}[1]{\oldparagraph{#1}\mbox{}}
\fi
\ifx\subparagraph\undefined\else
\let\oldsubparagraph\subparagraph
\renewcommand{\subparagraph}[1]{\oldsubparagraph{#1}\mbox{}}
\fi

\renewcommand{\familydefault}{\sfdefault}

\date{}

\begin{document}

\section{Mécanique}\label{muxe9canique}

\subsection{Thermodynamique}\label{thermodynamique}

\subsubsection{Système}\label{systuxe8me}

Un \textbf{système} est un corps ou un ensemble de corps de masse déterminée et délimitée dans l’espace.

Un système peut être \textbf{ouvert} (partie d’une canalisation) ou \textbf{fermé} (piston).

Un système \textbf{ouvert} échange de \textbf{l’énergie} et de la \textbf{matière} avec le milieu extérieur.

Un système est \textbf{homogène} s’il contient \textbf{une seule phase}. \\
Un système est \textbf{hétérogène} s’il contient plus d’une phase.

Un système est \textbf{isotherme} si sa \textbf{température} est \textbf{constante}. \\
Un système est \textbf{adiabatique} s’il \textbf{n’échange pas de chaleur} avec l’extérieur. \\
Un système est \textbf{diathermique} s’il \textbf{échange de la chaleur} avec
l’extérieur. \\

Un système est \textbf{isolé} s’il est \textbf{fermé}, \textbf{calorifugé} et
qu’\textbf{aucun travail} ne se produit dans le système. \\

Un \textbf{système isolé} n’échange \textbf{ni matière, ni chaleur, ni
travail} avec l’extérieur. \footnote{\url{https://fr.wikipedia.org/wiki/Système\_isolé}}


\subsubsection{État}\label{uxe9tat}

L’état thermodynamique d’un système est la description univoque de ce
système.

Cet état est fixé par un ensemble de variables d’état dont le choix est
arbitraire.

\subsubsection{Variables d’état}\label{variables-duxe9tat}

Les variables d’état peuvent être \textbf{extensives} ou
\textbf{intensives}.

\paragraph{Variables d’état
extensives}\label{variables-duxe9tat-extensives}

Une variable est \textbf{extensive} si sa valeur dépend de la taille du
système.

Par exemple : $m$, $V$, $E$

\paragraph{Variables d’état
intensives}\label{variables-duxe9tat-intensives}

Une variable est \textbf{intensive} si sa valeur ne dépend pas de la
taille du système.

Par exemple : $ T $, $ p $ ainsi que les grandeurs molaires
($ v $, $ V $, $ e $, $ E $)

\subsubsection{Fluide}\label{fluide}

Un \textbf{fluide} est un gaz ou un liquide.

Les \textbf{phases} possibles d’un \textbf{fluide} sont la phase
\textbf{gazeuse}, la phase \textbf{liquide} et la phase \textbf{solide}.

Un \textbf{cycle} consiste en une transformation qui ramène le fluide à
son état initial.

\subsubsection{Chaleur}\label{chaleur}

La \textbf{chaleur} est une forme de \textbf{travail}, qui correspond à
une augmentation ou une diminution de \textbf{l’agitation des
particules} constituant la matière.

\subsubsection{Travail}\label{travail}

Le \textbf{travail d’une force} est l’énergie fournie par cette force
lorsque son point d’application se déplace.

Le \textbf{travail} est souvent noté \textbf{W}.

Son unité est le \textbf{joules} (\textbf{J}).

\paragraph{Joule}\label{joule}

Le \textbf{joule} est le \textbf{travail} d’une force motrice
d’\textbf{un newton} dont le point d’application se déplace d’\textbf{un
mètre} dans la direction de la force.

$$ 1 J = 1 N \cdot m = \dfrac{ 1 kg \cdot m \cdot m }{ s } = 1 \cdot \dfrac { kg \cdot m^2}{ s }  $$

\begin{itemize}
\tightlist
\item
  $J$ = joule
\item
  $N$ = newton $ \left[ \dfrac{kg \cdot m}{s^2} \right] $
\item
  $m$ = mètre \\
\end{itemize}


Un \textbf{joule} est aussi l’énergie fournie par une puissance de
\textbf{1 Watt} pendant \textbf{1 seconde}.

$$ 1 J = 1 W \cdot s $$

Un \textbf{joule} est l’énergie requise pour élever la température d’un
gramme d’air d’un degré Celsius.

\paragraph{Newton}\label{newton}

Le \textbf{newton} est l’unité de mesure de la \textbf{force}. \footnote{\url{https://fr.wikipedia.org/wiki/Newton\_(unité)}}

Un \textbf{newton} est la force capable de communiquer à une masse de
\textbf{1kg} une accélération de \textbf{1m/s²}.

$$ 1 N = 1 \cdot \dfrac{kg \cdot m}{s^2} $$



\subsubsection{Intensives massiques}\label{intensives-massiques}

\paragraph{Volume massique}\label{volume-massique}

$$ v = V / m ~ [ m^3 / kg ] $$

\begin{itemize}
\tightlist
\item
  $V$ = volume ~~ \([m^3]\)
\item
  $m$ = masse ~~ \([kg]\) \\
\end{itemize}

Le volume massique d’un objet est l’inverse de sa masse volumique.

$$ v = \dfrac{V}{m} = \dfrac{1}{\rho} $$

\paragraph{Énergie massique}\label{uxe9nergie-massique}

$$ e = E / m ~ [J / kg]$$

\begin{itemize}
\tightlist
\item
  $E$ = énergie ~~ $ [J] $
\item
  $m$ = masse ~~ $ [kg] $
\end{itemize}

\subsubsection{Intensives molaires}\label{intensives-molaires}

\paragraph{Volume molaire}\label{volume-molaire}

$$ \Upsilon = V/M $$

\begin{itemize}
\tightlist
\item
  $V$ = volume ~~ $ [m^3] $
\item
  $M$ = masse molaire ~~ $ [moles] $
\end{itemize}

\paragraph{Énergie molaire}\label{uxe9nergie-molaire}

$$ \epsilon = E / M $$

\begin{itemize}
\tightlist
\item
  $E$ = énergie ~~ $ [J] $
\item
  $M$ = masse molaire ~~ $ [moles] $
\end{itemize}

\paragraph{Quantité de chaleur (formule différentielle)}\label{quantituxe9-de-chaleur}

$$ \delta ~ Q = m \cdot c \cdot \Delta T $$ 

\begin{itemize}
\tightlist
\item
  $Q$ = élément infinitésimal de chaleur à un moment donné
  ~~ $ [ J ] $
\item
  $m$ = masse ~~ $ [kg] $
\item
  $c$ = capacité thermique massique / chaleur massique
  ~~ $ \left[ J / kg \cdot K \right] $
\item
  $\Delta T$ = différence de température subie par le fluide à un
  moment ~~ $ [ K ] $
\end{itemize}

\paragraph{Puissance calorifique}\label{puissance-calorifique}

$$ \dot{Q} = \dfrac{\delta ~ Q}{dt} = m \cdot c \cdot \dfrac{dT}{dt} $$

\begin{itemize}
\tightlist
\item
  puissance de la chaleur par unité de temps
\item
  $m$ = masse ~~ $ [kg] $
\item
  $c$ = capacité thermique massique / chaleur massique
  ~~ $ \left[ J / kg \cdot K \right] $
\end{itemize}

\subsubsection{Transformation de
compression}\label{transformation-de-compression}

Une transformation est le passage d’un état à l’autre. \footnote{\url{https://fr.wikiversity.org/wiki/Transformations\_thermodynamiques/Transformations}}

Une \textbf{transformation} est \textbf{réversible} si le chemin pour
passer de l’état 1 à l’état 2 peut être pris dans l’autre sens. \\
La transformation doit se dérouler en une succession d’états d’équilibre.
Ce cas n’existe pas réellement, la réversibilité n’est jamais totale.

Une \textbf{transformation} est \textbf{irréversible} si le chemin pour
passer d’un état à l’autre ne peut être annulé.

Une \textbf{transformation} est \textbf{quasi-statique} lorsque la
réversibilité n’est pas totale mais suffisamment proche de l’équilibre,
par des écarts infinitésimaux.

\paragraph{Transformation isochore}\label{transformation-isochore}

Le \textbf{volume du système} reste constant durant la transformation.

Si le volume reste constant, on peut déduire que le travail à fournir
est nul.

La pression et la température peuvent varier.

\paragraph{Transformation isobare}\label{transformation-isobare}

La \textbf{pression du système} reste constante durant la
transformation.

\paragraph{Transformation isotherme}\label{transformation-isotherme}

La \textbf{chaleur du système} reste constante durant la transformation.

\paragraph{Transformation adiabatique}\label{transformation-adiabatique}

Aucun transfert thermique n’est effectué avec l’extérieur durant la transformation.

La chaleur ne reste pas forcément constante dans le système.

\subsection{Équilibre thermodynamique}\label{equilibre-thermodynamique}

Si le système est isolé du milieu extérieur et qu’aucune modification de ses variables d’état n’intervient, alors ce système est en équilibre thermodynamique.

\subsubsection{Principe 0 (zéro)}\label{principe-0-zuxe9ro}

« Lorsque 2 corps sont en équilibre thermique avec un 3e, ces 2 corps
sont aussi en équilibre entre eux ».

\subsubsection{Relation de la calorimétrie}\label{relation-de-la-calorimetrie}

La quantité de chaleur échangée est proportionnelle :

\begin{itemize}
\tightlist
\item
  à la masse ;
\item
  à l’écart de température ;
\item
  à la capacité thermique massique.
\end{itemize}

Pendant un changement de phase, la température ne varie pas. La chaleur
latente est utilisée durant le changement de phase.

$$ Q = m \cdot l $$

\begin{itemize}
\tightlist
\item
  $ Q $ = chaleur  ~~ $ [ Joules ] $
\item
  $ m $ = masse ~~ $ [kg] $
\item
  $ l $ = chaleur latente \\
\end{itemize}

La \textbf{chaleur latente} change l’état physique d’une matière. Par opposition à la chaleur sensible qui modifie la température d’une matière. \footnote{\url{http://www.energieplus-lesite.be/index.php?id=11244}}



\subsubsection{Loi de Fourier}\label{loi-fourier}

$$ \delta \vec{\dot{Q}} = -k \cdot S \cdot \vec{grad}~ T $$

\begin{itemize}
	\item
		$k$ = conduction thermique (conductivité) de l’élément
	\item
		$S$ = surface de contact / échange
	\item	
		$\vec{grad}~ T$ = vecteur de variation de la température
\end{itemize}



\subsection{Transfert de chaleur}\label{transfert-chaleur}

\subsubsection{Conduction}\label{conduction}

La conduction est la propagation de la chaleur de proche en proche d’un corps à l’autre par contact physique direct.


$$ \dot{Q} = k \cdot S \cdot \dfrac{\Delta T}{L} $$

\begin{itemize}
	\item 
		$ \dot{Q} $ = flux en mouvement ~~  $[Watt]$
	\item
		$ k $ = coefficient de conduction (conductivité) ~~ $[W/m \cdot K]$
	\item
		$ S $ = surface d’échange ~~ $[m^2]$
	\item
		$ \Delta T $ = différence de température ~~ $[\degree C]$ ou $[K]$
	\item
		$ L $ = $e$ = épaisseur ~~ $[mètres]$
\end{itemize}



$$ \dot{Q} = \dfrac{\Delta T}{R} $$

\begin{itemize}
	\item
		$ \dot{Q} $ = flux en mouvement ~~ $[Watt]$
	\item
		$ \Delta T $ = différence de température ~~ $[\degree C]$ ou $[K]$
	\item 
		$ R $ = résistance ~~ $[\Omega]$
\end{itemize}


%
\paragraph{Densité de flux}


$$ \dot{q} = \dfrac{\dot{Q}}{S} $$

\begin{itemize}
	\item 
		$ \dot{q} $ = densité de flux ~~ $[W/m^2]$
	\item 
		$ \dot{Q} $ = flux en mouvement ~~  $[Watt]$
\end{itemize}



\subsubsection{Convection}\label{convection}

La convection est l’échange de chaleur entre une surface et un fluide en mouvement.


$$ \dot{q} = h \cdot \Delta T $$

\begin{itemize}
	\item 
		$ \dot{q} $ = densité de flux ~~ $[W/m^2]$
	\item 
		$ h $ = coefficient de convection ~~  $[W/m^2 \cdot K]$
	\item
		$ \Delta T $ = différence de température ~~ $[\degree C]$ ou $[K]$
\end{itemize}



\subsubsection{Rayonnement}\label{rayonnement}





\subsubsection{Principe d’équivalence}\label{principe-equivalence}

Lors de toute transformation cyclique d’un système thermodynamique
fermé, \textbf{la somme des travaux mécaniques et des quantités de chaleur
échangées} entre le système et son environnement est \textbf{nulle}.

$$ \oint (\delta W + \delta Q) = 0 $$



\subsection{Premier principe de la thermodynamique}\label{premier-principe-thermo}

Au cours d’une transformation adiabatique, la \textbf{variation de l’énergie} d’un système est égale au \textbf{travail échangé} avec le milieu extérieur.

$$ \Delta E = \sum \delta W + \sum \delta Q = 0 $$


\newpage

\subsubsection{Rendement}

Le rendement mesure le rapport de l’énergie utile sur l’énergie consommée :

$$ \upeta = \dfrac{\textrm{énergie utile}}{\textrm{énergie consommée}} = \pm x \% $$

Ex: le rapport entre le travail utile fourni et la quantité de chaleur consommé.



\subsubsection{Efficacité}

Rapport de ce qui a été produit sur l’idéal.

Aussi appelé coefficient de performance \footnote{\url{https://fr.wikipedia.org/wiki/Coefficient_de_performance}}.

$$ \varepsilon = \dfrac{\textrm{produit}}{\textrm{idéal}} $$

Sur base de la proposition de Thomson/Joule, qui s’applique pour toute machine effectuant une transformation réversible (par exemple une machine de Carnot) :

$$ \left| \dfrac{\dot{Q}_{TH}}{\dot{Q}_{TB}} \right| \equiv \dfrac{T_H}{T_B}$$

avec \begin{itemize}
	\item 
		$\dot{Q}_{TH}$ le débit de chaleur absorbé ou rejeté à haute température ;
	\item
		$\dot{Q}_{TB}$ le débit de chaleur absorbé ou rejeté à basse température ;
	\item
		$T_H$ et $T_B$ les températures en $Kelvin$ ;
\end{itemize}

 on peut en déduire  :
 
 $$ \varepsilon 
 = \dfrac{Q_{fr}}{W}
 = \dfrac{1}{\dfrac{T_{ch}}{T_{fr}} - 1} $$
 
\paragraph{Démonstration :}

$$ \varepsilon 
= \dfrac{Q_{fr}}{W} 
= \dfrac{Q_{fr}}{Q_{ch} - Q_{fr}} 
= \dfrac{Q_{fr} \cdot \dfrac{1}{Q_{fr}}}{(Q_{ch} - Q_{fr}) \cdot \dfrac{1}{Q_{fr}}} 
= \dfrac{\dfrac{Q_{fr}}{Q_{fr}}}{\dfrac{Q_{ch}}{Q_{fr}} - \dfrac{Q_{fr}}{Q_{fr}}} 
= \dfrac{1}{\dfrac{Q_{ch}}{Q_{fr}} - 1}
= \dfrac{1}{\dfrac{T_{ch}}{T_{fr}} - 1} $$


\newpage


\subsubsection{Moteur de Carnot}

Le rendement du moteur de Carnot produit du travail en utilisant la chaleur injectée. \\

Un \textbf{rendement de 1} est \textbf{impossible} (monotherme).

$$ \upeta = \left| \dfrac{W}{Q_{ch} } \right| = \dfrac{| W |}{Q_{ch}}  = \dfrac{|Q_{ch}|}{Q_{ch}} - \dfrac{|Q_{fr}|}{Q_{ch}} =  1 - \dfrac{|Q_{fr}|}{Q_{ch}} $$


car d’après le \hyperref[principe-equivalence]{principe d’équivalence} \footnote{Voir page \pageref{principe-equivalence}}, $\delta W = -\delta Q$ donc $\delta W = -(Q_{ch} + Q_{fr})$, ce qui donne $\delta W = Q_{ch} - Q_{fr}$. \\


\subsubsection{Frigo de Carnot}

Le rendement du frigo de Carnot produit de la chaleur froide en utilisant le travail. \\

Un \textbf{rendement supérieur à 1} est \textbf{possible} (efficacité frigo).

$$ \upeta = \dfrac{Q_{fr}}{W} = \dfrac{1}{ \dfrac{|Q_{ch}|}{Q_{fr} } - 1 } $$



\subsubsection{Pompe à chaleur}

Une pompe à chaleur consomme du travail pour produire de la chaleur de chauffe. \\

Le \textbf{rendement} est \textbf{supérieur à 1} (coefficient de performance).

$$ \upeta = \dfrac{|Q_{ch}|}{W} = \dfrac{1}{ 1 - \dfrac{Q_{fr}}{ |Q_{ch}| } } $$



\subsection{Corollaires}\label{corollaires}

\subsubsection{Premier corollaire}

Un cycle moteur \textbf{irréversible} a nécessairement un \textbf{rendement inférieur} au cycle moteur \textbf{réversible}.

Cela s’applique seulement si les 2 cycles fonctionnent à partir de réservoirs de chaleur identiques, aux mêmes températures.


\subsubsection{Deuxième corollaire}

À condition d’utiliser des réservoirs de chaleur identiques, aux mêmes températures, tous les \textbf{moteurs réversibles} ont le \textbf{même rendement}.


\subsubsection{Troisième corollaire}

Le rendement d’un cycle moteur réversible ne dépend que de la température des réservoirs de chaleur.

$$ \upeta = 1 - \dfrac{|Q_{fr}|}{Q_{ch}} $$














\end{document}
